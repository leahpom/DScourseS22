\documentclass{article}

% Language setting
% Replace `english' with e.g. `spanish' to change the document language
\usepackage[english]{babel}

% Set page size and margins
% Replace `letterpaper' with`a4paper' for UK/EU standard size
\usepackage[letterpaper,top=2cm,bottom=2cm,left=3cm,right=3cm,marginparwidth=1.75cm]{geometry}

% Useful packages
\usepackage{amsmath}
\usepackage{graphicx}
\usepackage[colorlinks=true, allcolors=blue]{hyperref}

\title{Problem Set 1}
\author{Leah Pomerantz}

\begin{document}
\maketitle


\section{Interests in Economics and Data Science}

I'm very interested in learning more of the technical side of data science. I've taken several statistics classes while at OU, and a computer science class, but I still feel like I don't know very much about data science as a whole. My main goal for this class is to gain a deeper understanding of the non-statistical aspect of data science.  I feel as though I have an okay understanding of the statistics, but I'm so lost on the technical aspects. The most I've ever had to do was learn basic GIT terminal commands in one of my MATH stats classes a year ago, and we didn't need to use them again after spending about 45 minutes total (including pre-class work and a quiz) on the subject. I am very interested in learning tools as simple as using a supercomputer so I am better prepared for whatever PhD program I end up in.  I'm not quite sure what I'm planning on doing for my project yet. I may redo my project from econometrics, perhaps with more up-to-date data or trying to find data on COVID-19 vaccination rates.  After graduation, I plan on completing a PhD program in economics, and I think this class will increase my preparation levels for that. 

\section{Equation}

\begin{equation*}
    a^2 + b^2 = c^2
\end{equation*}


\end{document}