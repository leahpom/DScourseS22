\documentclass{article}

% Language setting
% Replace `english' with e.g. `spanish' to change the document language
\usepackage[english]{babel}

% Set page size and margins
% Replace `letterpaper' with`a4paper' for UK/EU standard size
\usepackage[letterpaper,top=2cm,bottom=2cm,left=3cm,right=3cm,marginparwidth=1.75cm]{geometry}

% Useful packages
\usepackage{amsmath,amsthm,amssymb}
\usepackage{enumitem}



\title{Problem Set 4}
\author{Leah Pomerantz}

\begin{document}
\maketitle


\section{Data Sources}

I think I'd be interested in scraping data from Twitter. I really enjoyed the Twitter scraping done in Dr. Wang's class, and it played nicely with R. I would be interested in looking at tweets using the hashtag "antisemitism." I think there are a lot of other topics I would be interested in, as the general concept of using Twitter data and text analysis fascinates me.

\section{Questions from Problem 6}

\begin{enumerate}[label=(\alph*)]
  \item 5.d \\
  mydf\$dates is a "character" type
  \item 6.7 \\
  "class(df1)" gives "tbl\_df"     "tbl" "data.frame" \\
  "class(df)" gives "tbl\_spark" "tbl\_sql"   "tbl\_lazy"  "tbl"  
  \item 6.8 \\
  The names of the columns are different by deliminator. For "df", there is a period between the two words and for "df1", there is a "\_" I think it might be because of how tibbles and Sparks may be read differently by the computer.
\end{enumerate}


\end{document}